\section{2.1 Amplifier with great amplification}
\subsection{theory}
An amplifier is given with, in figure: \ref{fig:circuit_1}, it is a non-inverting amplifier
with a gain of a 1001. The upper cutoff frequency is to be determined by the op amp itself
but the lower cutoff is to be determined by the RC-circuit on the output of the amplifier.

\begin{figure}[!htb]
    \centering
    \includegraphics[width=0.8\textwidth]{billeder/CIRCUIT_1.png}
    \caption{non-inverting amplifier with 1001 times gain}
    \label{fig:circuit_1}
\end{figure}

Since the filter is supposed to be designed for the human ear, the cutoff frequency should be somewhere
arund the 20Hz range. Choosing a suitable standard value capacitor, say 47nF, we can solve for the Required resistance
\begin{align}
     R_3 = \frac{1}{2 \pi C_1 R_3 } = 20~\text{Hz} &\Rightarrow \frac{1}{1880 \cdot nF \cdot Hz} = 169~\text{k} \Omega
\end{align}
we choose 180k$\Omega$, this yields a cut-off:
\begin{align}
    f_c = \frac{1}{2 \pi C_1 R_3} = 18.8~\text{Hz}
\end{align} 

Next is to determine the cut-off frequency of the operation amplifier. This is determined by the system gain and the \textit{open-loop gain, phase vs frequency}
which is figure: 2-21 in the MCP601 datasheet, this is also known as the \textit{gain-bandwidth product (GBP)}.
Which in this case is $GBP = 2.8~\text{MHz}$

As stated in the exercise, the gain of the system is 1001, this is determined by the feedback network:
\begin{align}
    A_N = 1 + \frac{R2}{R1} \Rightarrow 1 + \frac{100\text{k} \Omega}{100 \Omega} = 1001
\end{align}

thus the bandwidth of the amplifier is:
\begin{align}
    f_B = \frac{GBP}{A_N} \Rightarrow \frac{2.8~\text{MHz}}{1001} = 2.8~\text{KHz}
\end{align}

Thus this circuit is not suited as an audio amplifier, due to the fact that the cutoff is an order of magnitude below the upper human hearing range of 20KHz

To calculate the noise of the amplifier, we have to identify individual contributors to this nois
These are \textit{resistor noise},\textit{current noise} and the \textit{voltage noise}.
To calculate this, some of the factors are stated in the datasheet\\

From the datasheet, the voltage noise is $e_{nw} = 21\, \frac{\mathrm{nV}}{\sqrt{\mathrm{Hz}}}$
and the current noise is $i_{nw} = 0.6\, \frac{\mathrm{fA}}{\sqrt{\mathrm{Hz}}}$, these values are both at $f_{CE} = f_{CI} = 1\, \mathrm{KHz}$.

From the bandwidth defined by the output filter and the op amp itself, it is possible to create a brick-wall equivalent.
which acts as an analogy for the white noise, when we pass $\frac{1}{f}$ noise through a first-order filter.

We create this equivalent by having two frequencies which the noise is "contained" within
\begin{align}
    f_L &= f_C \cdot \frac{\pi}{2} = 29.5~\text{Hz}\\
    f_H &= f_B \cdot \frac{\pi}{2} = 4.4~\text{KHz}
\end{align}

The bandwidth provided by these two frequencies are also known as \textit{equivalent noise bandwidth}
\\
The noise voltage and current can then be calculated as
\begin{align}
    E_N &= e_{nw} \sqrt{ f_c e \ln\!\left(\frac{f_H}{f_L}\right) + f_H - f_L } = 0.061 \mathrm{mV}\\
    I_N &= i_{nw} \sqrt{ f_c e \ln\!\left(\frac{f_H}{f_L}\right) + f_H - f_L } = 1.26 \mathrm{pA}
\end{align} 

the final op amp voltage noise then gets amplified by the noise gain
\begin{align}
    V_{outN_V} = A_N \cdot E_N = 61 \mathrm{mV}
\end{align}

The current noise, is the current that is entering the inverting input, from a loop, using superposition, 
the voltage source in the output of the op-amp gets grounded,
meaning that the current source essentially sees two resistances in parallel \\

%insert picture of ground op amp output voltage souce, and the feedback network to show parallel resistances
\begin{center}
\begin{circuitikz}
    \draw 
    (0,0) node[ground]{}
    to[american current source, l=$I_N$] (0,3)
    coordinate (top);

    \draw 
    (top) to[short] (2,3)
    to[R=$R_1$] (2,0) node[ground]{};

    \draw 
    (top) to[short] (4,3)
    to[R=$R_2$] (4,0) node[ground]{};
\end{circuitikz}
\end{center}
The equivalent resistance $R_{EQ} = R1||R2 \Rightarrow (\frac{1}{100\cdot 10^3 \Omega} + \frac{1}{100\Omega})^{-1} = 99.9 \Omega$.\\
voltage from the current noise then becomes:
\begin{align}
    V_{outN_I} = I_N \cdot R_{EQ} \cdot A_N = 0.126 \mathrm{\mu V} 
\end{align}

The final contribution is the thermal- or \textit{johnsonnoise} from the feedback network and from the resistor in the filter:

\begin{align}
    V_{out_{Req}} &= \sqrt{4 k t \cdot R_{EQ} \cdot (F_H - F_L)} \cdot A_N = 2.701\mathrm{mV}\\ 
    V_{out_{Rfilter}} &= \sqrt{4 k t \cdot 180 \mathrm{k \Omega} \cdot (F_L - 0 \mathrm{Hz})}  = 0.306 \mathrm{\mu V}
\end{align}


The total RMS voltage then becomes the sum of all the sources:
\begin{align}
    V_{TotRMS} = \sqrt{V_{outN_V}^2 + V_{outN_I}^2 + V_{out_{Req}}^2 + V_{out_{Rfilter}}^2} = 61.02 \mathrm{mV}
\end{align}

\clearpage


\subsection{simulation}

\begin{figure}[!htb]
    \centering
    \includegraphics[width=0.7\textwidth]{billeder/CIRCUIT_1_SIM.png}
    \caption{siumlation of the first circuit configuration. }
    \label{fig:circuit1_simulation}
\end{figure}

\subsection{realisation}
Following the schematic and building the circuits lead to a measurement of $AC_{RMS} = 1.730\mathrm{mV}$.
To increase the accuracy of the measurement, the bench top multimeter was used, as it has better accuracy in AC-RMS scenarios. 
Whereas the accuracy of the oscilloscope is limited, due to the fact it integrates the signal over a shorter time period,
usualy only the entire "screen".