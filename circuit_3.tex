\section{2.3 circuit illustrating the effect of shotnoise(diode noise)}


\subsection{theory}
Optical receiver circuits, use transimpedance amplifiers to make an I/V conversion,
however this is prone to noise from the current source and parasitic elements of the diode, such as parasitic capaticance and resistance.

\begin{figure}[!htb]
    \centering
    \includegraphics[width=0.7\textwidth]{billeder/CIRCUIT_3.png}
    \caption{photo detector circuit, a transimpedance amplifier, filtered into a noninverting amplifier with a gain of 1001}
    \label{fig:circuit_3}
\end{figure}

Above is the overall circuit, which contains subcircuits, the transimpedance amplifier implemented by op amp U3, the high pass filter $C2R6$ and a 
non-inverting amplifier U4. This also means there are multiple noise bandwidths. For U3 the bandwidth is determined by
the high-pass filter for the lower limit aswell as the gain-bandwidth product for the upper limit.
As for U4, the lower limit is determined by the measurement equipment, for an analog discovery the AC RMS is 3Hz, and the upper 
limit is determined by U4's gain bandwidth product. As will be shown later, the bandwidth of the op amp is above 1KHz, thus
the value for $e_{nw} = 21 \VperRootHz{n} $
\\
Determine the noise gains:
Since the transimpedance amplifier, is mainly dominated by current noise, the voltage noise gain is set to 1
\begin{align}
    A_{NU3} = 1 
\end{align}

For op amp U4 the noise gain is determined by the closed loop gain:
\begin{align}
    A_{NU4} = 1 + \frac{R10}{R11} = 1001
\end{align}
This gain is also the same as the signal gain.

The upper and lower limit for U3's noise bandwidth:
\begin{align}
    f_{U3l} &= \frac{1}{2 \pi R6 C2} \cdot \frac{\pi}{2} = 24.178 \mathrm{Hz} \\
    f_{U3h} &= \frac{f_t}{A_{NU4}} \cdot \frac{\pi}{2} = 4.394 \mathrm{kHz}
\end{align}


The main noise component of noise from a photodiode is shot-noise. Its noise originating from charge carriers
when they cross a potential barrer such as a diode junction. It is defined as $\sqrt{2 q I_d}$
where $q = 1.602\cdot 10^{-19}\mathrm{C}$, is the electron charge.
\\ 
It is possible to create an equation relating voltage noise from current as a function of the diode current $I_d$
\begin{align}
    V_{Ndiode}(I_d) = R8 \cdot A_{NU3} \cdot A_{U4} \cdot \sqrt{2 q I_d \cdot(f_{U3h} - f_{U3h})}
\end{align}

Voltage noise from op amp U3:
\begin{align}
    V_{NU3} = e_{nw} \cdot A_{NU3} \cdot A_{U4} \cdot \sqrt{f_{ce} \cdot \ln\!\left(\frac{f_{U3h}}{f_{U3l}}\right) + f_{U3h} -f_{U3l}} = 2.057 \mathrm{mV}
\end{align}

Current noise from op amp U3:
\begin{align}
    V_{NIU3} = A_{NU3} \cdot A_{U4} \cdot i_{ni} \cdot R8 \sqrt{(f_{U3h} - f_{U3h})} = 3.97 \mathrm{\mu V}
\end{align}

Thermal noise from R8:
\begin{align}
    V_{NR8} = A_{NU3} \cdot A_{U4} \sqrt{4kt \cdot R8 \cdot (f_{U3h} - f_{U3h})} = 2.693 \mathrm{mV}
\end{align}

Noise bandwidth from U4: 
\begin{align}
    f_{U4l} = 3 \mathrm{Hz} \cdot \frac{\pi}{2} = 4.712 \mathrm{Hz}\\
    f_{U4h} = f_{U3h} = 4.934\mathrm{kHz}
\end{align}

Next is all the noise that is related to the op amp U4, starting with noise from R6, 
however this is limited from lowpass relationship
\begin{align}
    V_{NR6} = A_{U4} \cdot \sqrt{4kt \cdot R6 \cdot (f_{U3l} - f_{U4l})} = 123.239 \mathrm{\mu V}
\end{align}

Voltage noise from U4:
\begin{align}
    V_{NU4} = e_{nw} \cdot A_{NU4} \cdot A_{U4} \cdot \sqrt{f_{ce} \cdot \ln\!\left(\frac{f_{U4h}}{f_{U4l}}\right) + f_{U4h} - f_{U4l} } = 2.227 \mathrm{mV}
\end{align}

Current noise from U4 positive and negative inputs:
\begin{align}
    V_{NIU4+} &= A_{NU} \cdot i_{ni} \cdot R6 \cdot \sqrt{(f_{U4h} - f_{U4l})} = 1.87 \mathrm{\mu V}\\  
    V_{NIU4-} &= A_{NU} \cdot i_{ni} \cdot \frac{R10R11}{R10+R11} \cdot \sqrt{(f_{U4h} - f_{U4l})} = 3.975 \mathrm{nV} 
\end{align}


And lastly thermal noise from U4 feedback network:
\begin{align}
    V_{NR10R11} = A_{NU4} \cdot \sqrt{4kt \cdot \frac{R10R11}{R10+R11} \cdot (f_{U4h} - f_{U4l})} = 85.318 \mathrm{\mu V}
\end{align}

The individual terms can then be squared, rooted and summed together to get the entire noise 
contribution of the entire circuit.
This total noise is a function of the diode current 
\begin{align}
    V_{Ntotal}(I_d) = \sqrt{V_{Ndiode}(I_d)^2 + V_{NU3}^2 + 
    V_{NIU3}^2 + V_{NR8}^2 + V_{NR6}^2 + V_{NU4}^2 + 
    V_{NIU4+}^2 + V_{NIU4-}^2  + V_{NR10R11}^2 } 
\end{align}






\subsection{simulation}
\subsection{realisation}